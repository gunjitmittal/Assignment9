\let\negmedspace\undefined{}
\let\negthickspace\undefined{}

\documentclass{beamer}
\usepackage{amsthm}
 \usepackage{gensymb}
 \usepackage{polynom}
\usepackage{amssymb}
%
  \usepackage{stfloats}
\usepackage{bm} 
 \usepackage{longtable}
 \usepackage{enumitem}
 \usepackage{mathtools}
 \usepackage{tikz}
  %  \usepackage[breaklinks=true]{hyperref}
  \usepackage{listings}
\usepackage{color}                                            
\usepackage{array}                                            
\usepackage{longtable}                                        
\usepackage{calc}                                             
     \usepackage{multirow}                                         
     \usepackage{hhline}                                           
     \usepackage{ifthen}                                           
     \usepackage{lscape}     
\usetheme{CambridgeUS}
\DeclareMathOperator*{\Res}{Res}
\DeclareMathOperator*{\equals}{=}
\renewcommand\thesection{\arabic{section}}
\renewcommand\thesubsection{\thesection.\arabic{subsection}}
\renewcommand\thesubsubsection{\thesubsection.\arabic{subsubsection}}
% \renewcommand\thesectiondis{\arabic{section}}
% \renewcommand\thesubsectiondis{\thesectiondis.\arabic{subsection}}
% \renewcommand\thesubsubsectiondis{\thesubsectiondis.\arabic{subsubsection}}
\hyphenation{op-tical net-works semi-conduc-tor}
 \def\inputGnumericTable{}                                 %%
\lstset{ 
frame=single,
breaklines=true,
columns=fullflexible
}

% \newtheorem{theorem}{Theorem}[section]
% \newtheorem{problem}{Problem}
% \newtheorem{proposition}{Proposition}[section]
% \newtheorem{lemma}{Lemma}[section]
% \newtheorem{corollary}[theorem]{Corollary}
% \newtheorem{example}{Example}[section]
% \newtheorem{definition}[problem]{Definition}
\newcommand{\BEQA}{\begin{eqnarray}}
\newcommand{\EEQA}{\end{eqnarray}}
\newcommand{\define}{\stackrel{\triangle}{=}}
\newcommand*\circled[1]{\tikz[baseline= (char.base)]{
    \node[shape=circle,draw,inner sep=2pt] (char) {#1};}}
\bibliographystyle{IEEEtran}
\providecommand{\mbf}{\mathbf}
\providecommand{\pr}[1]{\ensuremath{\Pr\left(#1\right)}}
\providecommand{\qfunc}[1]{\ensuremath{Q\left(#1\right)}}
\providecommand{\sbrak}[1]{\ensuremath{{}\left[#1\right]}}
\providecommand{\lsbrak}[1]{\ensuremath{{}\left[#1\right.]}}
\providecommand{\rsbrak}[1]{\ensuremath{{}\left[#1\right.]}}
\providecommand{\brak}[1]{\ensuremath{\left(#1\right)}}
\providecommand{\lbrak}[1]{\ensuremath{\left(#1\right.)}
\providecommand{\rbrak}[1]{\ensuremath{\left[#1\right.]}}}
\providecommand{\cbrak}[1]{\ensuremath{\left\{#1\right\}}}
\providecommand{\lcbrak}[1]{\ensuremath{\left\{#1\right.}}
\providecommand{\rcbrak}[1]{\ensuremath{\left.#1\right\}}}
\theoremstyle{remark}
\newtheorem{rem}{Remark}
\newcommand{\sgn}{\mathop{\mathrm{sgn}}}
\providecommand{\abs}[1]{\left\vert#1\right\vert}
\providecommand{\res}[1]{\Res\displaylimits_{#1}} 
\providecommand{\norm}[1]{\left\lVert#1\right\rVert}
\providecommand{\mtx}[1]{\mathbf{#1}}
\providecommand{\mean}[1]{E\left[ #1 \right]}
\providecommand{\fourier}{\overset{\mathcal{F}}{ \rightleftharpoons}}
\providecommand{\system}{\overset{\mathcal{H}}{ \longleftrightarrow}}
% \newcommand{\solution}{\noindent \textbf{Solution: }}
\newcommand{\cosec}{\,\text{cosec}\,}
\newcommand*{\permcomb}[4][0mu]{{{}^{#3}\mkern#1#2_{#4}}}
\newcommand*{\perm}[1][-3mu]{\permcomb[#1]{P}}
\newcommand*{\comb}[1][-1mu]{\permcomb[#1]{C}}
\renewcommand{\thetable}{\arabic{table}} 
\providecommand{\dec}[2]{\ensuremath{\overset{#1}{\underset{#2}{\gtrless}}}}
\newcommand{\myvec}[1]{\ensuremath{\begin{pmatrix}#1\end{pmatrix}}}
\newcommand{\mydet}[1]{\ensuremath{\begin{vmatrix}#1\end{vmatrix}}}
\numberwithin{equation}{section}
\numberwithin{figure}{section}
\numberwithin{table}{section}
\makeatletter
\@addtoreset{figure}{problem}
\makeatother
\let\StandardTheFigure\thefigure{}
\let\vec\mathbf{}
\def\putbox#1#2#3{\makebox[0in][l]{\makebox[#1][l]{}\raisebox{\baselineskip}[0in][0in]{\raisebox{#2}[0in][0in]{#3}}}}
     \def\rightbox#1{\makebox[0in][r]{#1}}
     \def\centbox#1{\makebox[0in]{#1}}
     \def\topbox#1{\raisebox{-\baselineskip}[0in][0in]{#1}}
     \def\midbox#1{\raisebox{-0.5\baselineskip}[0in][0in]{#1}}
\vspace{3cm}
\title{Assignment 9 Papoulis ex 6.63}
\author{Gunjit Mittal (AI21BTECH11011)}
\date{\today}
\logo{\large \LaTeX}
\begin{document} 
\begin{frame}
  \titlepage{}
\end{frame}
\logo{}
\begin{frame}{Outline}
  \tableofcontents
\end{frame}
% Download all python codes from 
% \begin{lstlisting}
% https://github.com/GunjitMittal/Assignment6/tree/main/Assignment6/code
% \end{lstlisting}     
% Download all latex codes from 
% \begin{lstlisting}
% https://github.com/GunjitMittal/Assignment6/tree/main/Assignment6 
% \end{lstlisting} 
\section{Question}
\begin{frame}{Question}
For any two random variables x and y, let $\sigma^2_x = Var\cbrak{x},\sigma^2_y = Var\cbrak{y} and \sigma^2_{x+y} = Var\cbrak{x+y}$\\
(a) show that
\begin{align*}
    \frac{\sigma_{x+y}}{\sigma_x+\sigma_y}\leq1
\end{align*}
(b)More generally, show that for p $\geq$ 1
\begin{align*}
  \frac{\cbrak{{E(|x+y|}^p)}^{1/p}}{\cbrak{{E(|x|}^p)}^{1/p}+\cbrak{{E(|y|}^p)}^{1/p}} \leq 1
\end{align*}
\end{frame}
\section{Solution (a)} 
\begin{frame}{Solution (a)}
For any two two random variables X and Y we have 
\begin{align*}
  &\sigma_{X+Y} = Var(X+Y) = E\sbrak{\cbrak{\brak{X-\mu_X}+\brak{Y-\mu_Y}}^2}\\
  &~~~~~~~~~~~~=Var(X)+Var(Y)+2Cov(X,Y) = \sigma^2_X + \sigma^2_Y +2 \sigma_X\sigma_Y\rho_{XY}\\
  &~~~~~~~~~~~~~\leq \brak{\sigma_X+\sigma_Y}^2
\end{align*}
Since $|\rho_{XY}|\leq1$.Thus
\begin{align*}
  \sigma_{X+Y}\leq \sigma_X + \sigma_Y,
\end{align*}
and hence it easily follows that
\begin{align*}
  \frac{\sigma_{X+Y}}{\sigma_X+\sigma_Y}\leq1.
\end{align*}
\end{frame}
\section{Solution (b)}
\begin{frame}{Solution (b)}  
We shall prove this result in three parts by making use of Holder's inequality.\\
(i)\textbf{ Holder's inequality:} The function $\log x$ is concave, for $0<\alpha<1$, and hence we have
\begin{align*}
  \log \sbrak{\alpha x_1+(1-\alpha)x_2} \geq \alpha \log x_1+(1-\alpha)\log x_2
\end{align*}
or
\begin{align}
  x_1^\alpha x_2^{1-\alpha} \leq \alpha x_1 + (1-\alpha)x_2,~~~~ 0<\alpha<1
  \label{eq:eq1}
\end{align}
Let
\begin{align}
  x_1 = |x|^p,~~~\alpha = \frac{1}{p}, \text{so that}~~ 1-\alpha = 1-\frac{1}{p} \triangleq \frac{1}{q},~~~~x_2 = |y|^q
  \label{eq:eq2}
\end{align}
\end{frame}
\begin{frame}
  so that \eqref{eq:eq1} becomes
  \begin{align}
    |xy| \leq \frac{|x|^p}{p} + \frac{|y|^q}{q}, p>1,
    \label{eq:eq3}
  \end{align}
  the Holder's inequality. From \eqref{eq:eq2}, note that
  \begin{align}
    \frac{1}{p} + \frac{1}{q} = 1, ~~~ p>1, ~~~ q>1
    \label{eq:eq4}
  \end{align}
  (ii) Define
  \begin{align*}
    x = X\brak{E\cbrak{|X|^p}}^{-1/p},~~~~ y = Y\brak{E\cbrak{|Y|^q}}^{-1/q}
  \end{align*}
  where p and q are as in \eqref{eq:eq4}. Substituting these into the Holder's inequality in \eqref{eq:eq3}, we get
  \begin{align}
    |XY| \leq p^{-1}|X|^p\brak{E\cbrak{|X|^p}}^{1/p-1}\brak{E\cbrak{|Y|^q}}^{1/q}\nonumber\\
    +q^{-1}|Y|^q\brak{E\cbrak{|Y|^q}}^{1/q-1}\brak{E\cbrak{|X|^p}}^{1/p}
    \label{eq:eq5}
  \end{align}
\end{frame}
\begin{frame}
  Taking expected values on both sides of \eqref{eq:eq5}, we get
  \begin{align}
    E\cbrak{|XY|}\leq\brak{E\cbrak{|X|^p}}^{1/p}\brak{E\cbrak{|Y|^q}}^{1/q}
    \label{eq:eq6}
  \end{align}
  which represents the generalization of the Cauchy-Schwarz inequality.\\
  (Note p=q=2 corresponds to Cauchy-Schwarz inequality)
  (iii) To prove the desired inequality, notice that
  \begin{align*}
    &|X+Y|^p = |X+Y||X+Y|^{p-1}\\
    &~~~~~~~~~~~\leq|X||X+Y|^{p-1}+|Y||X+Y|^{p-1},~~~~p>1
  \end{align*}
  and taking the expected values on both sides we get
  \begin{align}
    E\cbrak{|X+Y|^p}\leq E\cbrak{|X||X+Y|^{p-1}}+ E\cbrak{|Y||X+Y|^{p-1}}
    \label{eq:eq7}
  \end{align}
  Applying \eqref{eq:eq6} to each term on the right side of \eqref{eq:eq7} we get
  \begin{align}
    E\cbrak{|X||X+Y|^{p-1}}\leq  \brak{E\cbrak{|X|^{p}}}^{1/p}\brak{E\cbrak{|X+Y|^{(p-1)q}}}^{1/q}
    \label{eq:eq8}
  \end{align}
\end{frame}
\begin{frame}
  and
  \begin{align}
    E\cbrak{|Y||X+Y|^{p-1}}\leq  \brak{E\cbrak{|Y|^{p}}}^{1/p}\brak{E\cbrak{|X+Y|^{(p-1)q}}}^{1/q}
    \label{eq:eq9}
  \end{align}
  Using \eqref{eq:eq8} and \eqref{eq:eq9} together with $(p-1)q = p$ in \eqref{eq:eq7} we get
  \begin{align*}
    E\cbrak{|X+Y|^p}\leq \sbrak{\brak{E\cbrak{|X|^{p}}}^{1/p}+\brak{E\cbrak{|Y|^{p}}}^{1/p}}.\brak{E\cbrak{|X+Y|^{(p-1)q}}}^{1/q}
  \end{align*}
  or for $p>1$
  \begin{align*}
    \brak{E\cbrak{|X+Y|^p}}^{1/p}\leq \brak{E\cbrak{|X|^{p}}}^{1/p}+\brak{E\cbrak{|Y|^{p}}}^{1/p}
  \end{align*}
  the desired inequality. Since $p=1$ follows trivially, we get
  \begin{align*}
    \frac{ \brak{E\cbrak{|X+Y|^p}}^{1/p}}{\brak{E\cbrak{|X|^{p}}}^{1/p}+\brak{E\cbrak{|Y|^{p}}}^{1/p}}\leq 1, ~~~~ p\geq 1
  \end{align*}
\end{frame}
\end{document}     